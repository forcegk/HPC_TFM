%%%%%%%%%%%%%%%%%%%%%%%%%%%%%%%%%%%%%%%%%%%%%%%%%%%%%%%%%%%%%%%%%%%%%%%%%%%%%%%%
\begin{abstract}\thispagestyle{empty}
  En un mundo donde la inteligencia artificial es cada vez más atractiva para el público general, con modelos que realizan tareas como la conducción autónoma, se puede ver cómo estas redes neuronales, que llevan evolucionando a enorme velocidad los últimos años, son cada vez más útiles para el público general, generándose así una gran demanda computacional y energética tanto en los dispositivos más grandes, como los automóviles, como en los más pequeños, como pulseras inteligentes. Debido a esta razón, cada vez son más necesarias arquitecturas de altas prestaciones y elevada eficiencia energética para optimizar su rendimiento.

  En este Trabajo Fin de Máster se documenta el análisis, mejora y medida del rendimiento de ejecución de modelos de inteligencia artificial con redes \textit{feed-forward} dispersas en fase de inferencia. Mediante el uso de una herramienta para generación de código C desde modelos TensorFlow se comparan diferentes aproximaciones a la multiplicación de matrices, clave en inteligencia artificial, y se allana el camino hacia subsecuentes optimizaciones.

  \vspace*{25pt}
  \begin{segundoresumo}

  In a world where artificial intelligence is becoming increasingly attractive to the general public, with models performing tasks such as autonomous driving, we can see how these neural networks, which have been evolving at enormous speed in recent years, are becoming increasingly useful to the general public, thus generating a high computational and energy demand in both larger devices, such as cars, and smaller ones, such as smart bands. For this reason, high performance and efficient architectures are increasingly necessary to optimize their performance.

  This dissertation documents the analysis, improvement and performance measurement of artificial intelligence model execution with sparse \textit{feed-forward} networks in the inference phase. By using a tool for C code generation from TensorFlow models, we compare different approaches to matrix multiplication, key in artificial intelligence, and pave the way to subsequent optimizations.

  \end{segundoresumo}
\vspace*{25pt}
\begin{multicols}{2}
\begin{description}
\item [\palabraschaveprincipal:] \mbox{} \\[-20pt]
  \begin{itemize}
    \item Supercomputación
    \item Clúster
    \item Raspberry Pi
    \item MPI
    \item Divulgación
    \item Docencia
    \item Benchmarks
    \item Arch Linux
  \end{itemize}
\end{description}
\begin{description}
\item [\palabraschavesecundaria:] \mbox{} \\[-20pt]
  \begin{itemize}
    \item High Performance Computing
    \item Cluster
    \item Raspberry Pi
    \item MPI
    \item Pop-science
    \item Teaching
    \item Benchmarks
    \item Arch Linux
  \end{itemize}
\end{description}
\end{multicols}
\end{abstract}
%%%%%%%%%%%%%%%%%%%%%%%%%%%%%%%%%%%%%%%%%%%%%%%%%%%%%%%%%%%%%%%%%%%%%%%%%%%%%%%%
%%%%%%%%%%%%%%%%%%%%%%%%%%%%%%%%%%%%%%%%%%%%%%%%%%%%%%%%%%%%%%%%%%%%%%%%%%%%%%%%
% Obxectivo: Lista de termos empregados no documento,                          %
%            xunto cos seus respectivos significados.                          %
%%%%%%%%%%%%%%%%%%%%%%%%%%%%%%%%%%%%%%%%%%%%%%%%%%%%%%%%%%%%%%%%%%%%%%%%%%%%%%%%

\newglossaryentry{rootfs}{
  name=rootfs,
  description={Sistema de ficheros raíz, que contiene al resto de ficheros y directorios y por tanto permite que se arranque un sistema operativo desde los contenidos del mismo}
}

\newglossaryentry{backend}{
  name=backend,
  description={En desarrollo de software es la parte que se encarga funcionamiento de la lógica de programa. Esta lógica por lo general no es visible para el usuario (por ejemplo el acceso a bajo nivel, así como lectura y escritura de datos), por lo que se dice que está en la ``parte-de-atrás'' o back-end, y el usuario no debe interactuar con ella}
}

\newglossaryentry{overclock}{
  name=overclock,
  description={El overclocking es una práctica, habitualmente realizada por usuarios entusiastas, que consiste en elevar la frecuencia de reloj de un componente por encima de las especificaciones que dictamina el fabricante}
}
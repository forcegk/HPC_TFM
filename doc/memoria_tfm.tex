%%%%%%%%%%%%%%%%%%%%%%%%%%%%%%%%%%%%%%%%%%%%%%%%%%%%%%%%%%%%%%%%%%%%%%%%%%%%%%%%
% Preámbulo                                                                    %
%%%%%%%%%%%%%%%%%%%%%%%%%%%%%%%%%%%%%%%%%%%%%%%%%%%%%%%%%%%%%%%%%%%%%%%%%%%%%%%%

\documentclass[11pt,a4paper,titlepage,twoside,openright,openbib,spanish]{report}

%%% RELACIÓN DE VARIABLES A PERSONALIZAR %%%
\def\lingua{esp} % descomenta esta liña se redactarás a memoria en español
\def\nome{Alonso Rodríguez Iglesias}                             % substitúe aquí o teu nome
\def\nomedirectorA{Gabriel Rodríguez Álvarez}             % substitúe aquí o nome de quen dirixe
\def\nomedirectorB{Juan Touriño Domínguez}             % substitúe aquí o nome de quen dirixe
\def\titulo{Análisis del rendimiento de la inferencia de redes de aprendizaje profundo en arquitecturas de altas prestaciones} % substitúe aquí o título do teu TFM
% \def\mencion{ENXEÑARÍA DE COMPUTADORES}

\def\renomearcadros{si} % descomenta esta liña se redactas a memoria en español e prefires que
                         % os "cuadros" e o "índice de cuadros" se renomeen
                         % a "tablas" e "índice de tablas" respectivamente

\usepackage{estilo_tfm}

% Lista de paquetes potencialmente interesantes (uso baixo demanda)

\usepackage[all]{nowidow}
% \usepackage{alltt}       % proporciona o entorno alltt, semellante a verbatim pero que respecta comandos
% \usepackage{enumitem}    % permite personalizar os entornos de lista
% \usepackage{eurofont}    % proporciona o comando \euro
\usepackage{eurosym}
\usepackage{float}       % permite máis opcións para controlar obxectos flotantes (táboas, figuras)
\usepackage{hhline}      % permie personalizar as liñas horizontais en arrays e táboas
% \usepackage{longtable}   % permite construir táboas que ocupan máis dunha páxina
\usepackage{lscape}      % permite colocar partes do documento en orientación apaisada
% \usepackage{moreverb}    % permite personalizar o entorno verbatim
\usepackage{multirow}    % permite crear celdas que ocupan varias filas da mesma táboa
\usepackage{pdfpages}    % permite insertar ficheiros en PDF no documento
\usepackage{rotating}    % permite diferentes tipos de rotacións para figuras e táboas
% \usepackage{subcaption}  % permite a inclusión de varias subfiguras nunha figura
% \usepackage{tabu}        % permite táboas flexibles
% \usepackage{tabularx}    % permite táboas con columnas de anchura determinada
\usepackage{pdflscape}
\usepackage{translator}
\usepackage{placeins}
\usepackage{tablefootnote}

\usepackage[scale=MatchLowercase]{sourcecodepro}
\usepackage{pgfplots}
\usepackage{pgfgantt}
\newganttlinktype{F-S}{
  \ganttsetstartanchor{east}
  \ganttsetendanchor{west}
  \draw [/pgfgantt/link] (\xLeft - 0.2,\yUpper) -- (\xRight + 0.2, \yLower);
}

\newganttlinktype{F_S}{
  \ganttsetstartanchor{east}
  \ganttsetendanchor{west}
  \draw [/pgfgantt/link] (\xLeft - 0.2,\yUpper) |- (\xRight + 0.2, \yLower);
}

\newganttlinktype{S-S}{
  \ganttsetstartanchor{west}
  \ganttsetendanchor{west}
  \draw [/pgfgantt/link] (\xLeft + 0.2,\yUpper) |- (\xRight, \yLower);
}

\newganttlinktype{F-F}{
  \ganttsetstartanchor{east}
  \ganttsetendanchor{east}
  \draw [/pgfgantt/link] (\xLeft - 0.2 ,\yUpper) -| (\xRight - 0.2, \yLower);
}

\newganttlinktype{F-S}{
  \ganttsetstartanchor{east}
  \ganttsetendanchor{west}
  \draw [/pgfgantt/link] (\xLeft - 0.2,\yUpper) -| (\xRight + 0.2, \yLower);
}
\usepackage{pgfplotstable}
\pgfplotsset{compat = newest}
\usepackage{wrapfig}
\usepackage{etoolbox}
\BeforeBeginEnvironment{wrapfigure}{\setlength{\intextsep}{0pt}}
\usepackage{svg}
\usepackage{adjustbox}
\newcommand{\vasymptote}[2][]{
    \draw [thick,ficblue,densely dashed,#1] ({rel axis cs:0,0} -| {axis cs:#2,0}) -- ({rel axis cs:0,1} -| {axis cs:#2,0});
}
\pgfplotsset{every axis/.append style={
        scaled y ticks = false, 
        scaled x ticks = false, 
        y tick label style={/pgf/number format/.cd, fixed},
        x tick label style={/pgf/number format/.cd, fixed}
    }
}

%%%%%%%%%%%%%%%%%%%%%%%%%%%%%%%%%%%%%%%%%%%%%%%%%%%%%%%%%%%%%%%%%%%%%%%%%%%%%%%%
% Corpo                                                                        %
%%%%%%%%%%%%%%%%%%%%%%%%%%%%%%%%%%%%%%%%%%%%%%%%%%%%%%%%%%%%%%%%%%%%%%%%%%%%%%%%

\begin{document}

%%%%%%%%%%%%%%%%%%%%%%%%%%%%%%%%%%%%%%%%
% Preliminares do documento            %
%%%%%%%%%%%%%%%%%%%%%%%%%%%%%%%%%%%%%%%%

\begin{titlepage}
  
  \hspace*{128pt}
  \textcolor{udcgray}{{\fontencoding{T1}\fontfamily{phv}\selectfont Departamento de Ingeniería de Computadores}}\\[-2pt]
  \hspace*{145pt}
  \textcolor{udcpink}{{\fontencoding{T1}\fontfamily{phv}\selectfont Facultad de Informática de A Coruña}}\\[-32pt]

  \begin{center}
    \includegraphics[scale=0.3]{img/udc.png}\\[35pt]

    {\large TRABAJO FIN DE MÁSTER \\
            MÁSTER INTERUNIVERSITARIO EN \\
            COMPUTACIÓN DE ALTAS PRESTACIONES } \\[100pt]
    
    \begin{huge}
      \begin{spacing}{1.3}
        \bfseries \titulo
      \end{spacing}
    \end{huge}
  \end{center}
  
  \vfill
  
  \begin{flushright}
    {\large
    \begin{tabular}{ll}
      {\bf Estudiante:} & \nome \\
      {\bf Directores:} & \nomedirectorA \\ % COPIA E PEGA ESTA LIÑA MÁIS VECES SE O PRECISAS
                        & \nomedirectorB \\ % COPIA E PEGA ESTA LIÑA MÁIS VECES SE O PRECISAS
    \end{tabular}}
  \end{flushright}
  \rightline{A Coruña, \today}
\end{titlepage}

\paxinaenbranco
\begin{flushright}
\dedicatoria{A todas las personas que me apoyaron en los buenos y los malos momentos.\\Gracias a ellos hoy soy quien soy.}
\end{flushright}
\paxinaenbranco
\paxinaenbranco
\begin{agradecementos}
A Gabriel y Juan por darme esta oportunidad, a mis amigos, que han hecho un poco más sobrellevable este (otra vez) atípico año, y que esperemos sea el último, no solo apoyándome siempre en mis objetivos, sino también personalmente, y especialmente a Xián por su inestimable ayuda domando la pitón.

Muchas gracias.

\begin{flushright}
Alonso
\end{flushright}
\end{agradecementos}
% Ugly hack to fix random pagenumber %%%%%%
\pagenumbering{gobble}
\pagestyle{empty}
%%%%%%%%%%%%%%%%%%%%%%%%%%%%%%%%%%%%%%%%%%%
\paxinaenbranco
%%%%%%%%%%%%%%%%%%%%%%%%%%%%%%%%%%%%%%%%%%%%%%%%%%%%%%%%%%%%%%%%%%%%%%%%%%%%%%%%
\begin{abstract}\thispagestyle{empty}
  En un mundo donde la inteligencia artificial es cada vez más atractiva para el público general, con modelos que realizan tareas como la conducción autónoma, se puede ver cómo estas redes neuronales, que llevan evolucionando a enorme velocidad los últimos años, son cada vez más útiles para el público general, generándose así una gran demanda computacional y energética tanto en los dispositivos más grandes, como los automóviles, como en los más pequeños, como pulseras inteligentes. Debido a esta razón, cada vez son más necesarias arquitecturas de altas prestaciones y elevada eficiencia energética para optimizar su rendimiento.

  En este Trabajo Fin de Máster se documenta el análisis, mejora y medida del rendimiento de ejecución de modelos de inteligencia artificial con redes \textit{feed-forward} dispersas en fase de inferencia. Mediante el uso de una herramienta para generación de código C desde modelos TensorFlow se comparan diferentes aproximaciones a la multiplicación de matrices, clave en inteligencia artificial, y se allana el camino hacia subsecuentes optimizaciones.

  \vspace*{25pt}
  \begin{segundoresumo}

  In a world where artificial intelligence is becoming increasingly attractive to the general public, with models performing tasks such as autonomous driving, we can see how these neural networks, which have been evolving at enormous speed in recent years, are becoming increasingly useful to the general public, thus generating a high computational and energy demand in both larger devices, such as cars, and smaller ones, such as smart bands. For this reason, high performance and efficient architectures are increasingly necessary to optimize their performance.

  This dissertation documents the analysis, improvement and performance measurement of artificial intelligence model execution with sparse \textit{feed-forward} networks in the inference phase. By using a tool for C code generation from TensorFlow models, we compare different approaches to matrix multiplication, key in artificial intelligence, and pave the way to subsequent optimizations.

  \end{segundoresumo}
\vspace*{25pt}
\begin{multicols}{2}
\begin{description}
\item [\palabraschaveprincipal:] \mbox{} \\[-20pt]
  \begin{itemize}
    \item Supercomputación
    \item Clúster
    \item Raspberry Pi
    \item MPI
    \item Divulgación
    \item Docencia
    \item Benchmarks
    \item Arch Linux
  \end{itemize}
\end{description}
\begin{description}
\item [\palabraschavesecundaria:] \mbox{} \\[-20pt]
  \begin{itemize}
    \item High Performance Computing
    \item Cluster
    \item Raspberry Pi
    \item MPI
    \item Pop-science
    \item Teaching
    \item Benchmarks
    \item Arch Linux
  \end{itemize}
\end{description}
\end{multicols}
\end{abstract}
%%%%%%%%%%%%%%%%%%%%%%%%%%%%%%%%%%%%%%%%%%%%%%%%%%%%%%%%%%%%%%%%%%%%%%%%%%%%%%%%
%\paxinaenbranco
%Restore thinges %%%
\pagestyle{fancy}
%%%%%%%%%%%%%%%%%%%%

\pagenumbering{roman}
\setcounter{page}{1}
\bstctlcite{IEEEexample:BSTcontrol}

\tableofcontents
\listoffigures
\listoftables
\cleardoublepage

\pagenumbering{arabic}
\setcounter{page}{1}

%%%%%%%%%%%%%%%%%%%%%%%%%%%%%%%%%%%%%%%%
% Capítulos                            %
%%%%%%%%%%%%%%%%%%%%%%%%%%%%%%%%%%%%%%%%

\chapter{Introducción}
\label{chap:introducion}

\lettrine{E}{n} este capítulo se expondrá qué motivó este \acrlong{tfm} y sus objetivos, así como qué estructura seguirá la memoria del mismo.

\section{Motivación}
\label{sec:motivacion}
La computación de altas prestaciones (\acrshort{hpc}, o \acrlong{hpc}) y la \acrlong{ia} (\acrshort{ia}) están de moda. A pesar de que la primera es parte necesaria de muchos modelos complejos, que se mencionan más adelante en este trabajo, debido a la alta demanda de potencia de cómputo en la fase de entrenamiento de los mismos, es cierto que la inteligencia artificial es la más espectacular y tangible de cara al público.
Estos modelos de redes neuronales están evolucionando a enorme velocidad, y con ello sus posibles aplicaciones presentes y futuras, y debido a esta fuerte entrada de la \acrshort{ia} en lo \textit{mainstream}, son cada vez más necesarias soluciones de bajo consumo en una economía de escala.

\section{Objetivos}
\label{sec:objetivos}
Este trabajo supone un paso más en la nueva línea de investigación que comencé este curso académico, y por tanto no es de un muy elevado nivel debido a la falta de experiencia previa en este campo. Sin embargo es cierto que para esta memoria se han intentado incorporar conocimientos adquiridos en varias asignaturas cursadas durante este año en el Máster en Computación de Altas Prestaciones.



\section{Estructura}
\label{sec:estructura}
loremipsum
\chapter{Conceptos básicos}
\label{chap:conceptos_basicos}

\lettrine{P}{ara} una correcta comprensión de los objetivos, tanto a corto como a largo plazo, de este trabajo, es necesario entender los conceptos básicos sobre los que se apoya esta línea de investigación.

\section{Redes neuronales}
\label{sec:redes_neuronales}
Las redes neuronales constituyen la base de la mayoría de últimos avances en el campo de la inteligencia artificial, y a pesar de la enorme diversidad en \textit{layouts} de capas, neuronas, funciones de transferencia, etc, la realidad es que el álgebra lineal y la multiplicación de matrices son parte fundamental e imprescindible para la ejecución de las mismas. \cite[Figure 3.4]{deep_learning_for_computer_architects}

A pesar de que en la \acrshort{poc} (\acrlong{poc}) expuesta más adelante en este trabajo se trabaja únicamente con redes \textit{Feed-Forward}, sería naíf pensar que solamente existen estas arquitecturas de redes neuronales. Ejemplos a destacar serían las redes convolucionales, empleadas principalmente para el procesado de imágenes, las basadas en modelos secuenciales o en transformers, que están revolucionando el mundo de la inteligencia artificial mediante modelos de procesamiento del lenguaje o de generación de imágenes, así como muchas otras como las redes GAN, o las basadas en Encoder-Decoder.

Estas arquitecturas son llamadas de redes neuronales, a pesar de que como es evidente, una neurona no puede, como tal ``ejecutar'' una convolución. Más bien, estas nuevas arquitecturas se basan en la modelación de operaciones matemáticas complejas en pasos discretos como, por ejemplo, la convolución que se menciona previamente.

\subsection{Estructura de una red neuronal feed-forward}
\label{ssec:estructura_red_neuronal_ff}
Las redes neuronales \textit{feed-forward} se componen de varias capas de neuronas que toman una o más entradas, realizan la suma de todas ellas, suman un \textit{bias}, y finalmente aplican una función de transferencia no lineal.

La capa que toma los datos de entrada se llama capa \textit{input} o de entrada, y la capa que expulsa los datos de salida es la capa \textit{output} o de salida. Las capas que realizan transformaciones intermedias se denominan capas ocultas, o \textit{hidden layers}.

En la Figura \ref{fig:dense_nn_sample}\footnote{Imagen generada con la herramienta NN-SVG, disponible en \url{https://alexlenail.me/NN-SVG}} se puede ver un diagrama de una red neuronal Feed-Forward densa. Esto es, una red neuronal en la que cada neurona de una capa está conectada con todas las neuronas de la capa a continuación. Esto se traduce en que la matriz de pesos que representa cada capa es una matriz densa, o dicho de otra manera, que no contiene pesos con valor 0.

\begin{figure}[h!]
    \centering
    \vspace*{0.5cm}
    \def\svgwidth{0.85\textwidth}
    \input{pdf_tex/dense_nn/dense_nn_svgnn.pdf_tex}
    \caption{Red neuronal densa Feed-Forward}
    \label{fig:dense_nn_sample}
\end{figure}

Por el contrario, y como se puede ver en la Figura \ref{fig:sparse_nn_sample}, existen las redes neuronales dispersas, en las que una cantidad significativa de los pesos tienen valor cero. Estas redes se pueden obtener de múltiples formas, y sus objetivos son variados, a pesar de que el principal objetivo es reducir el tamaño del modelo.

Uno de las posibles métodos para la obtención de una red neuronal dispersa (\textit{sparse}) es mediante el podado o \textit{pruning}. Este método, sorprendentemente similar al empleado por neurocirujanos durante operaciones a cerebro abierto, consiste en ir cortando de forma más o menos arbitraria (en función del algoritmo) conexiones y observando cómo varía la salida. En caso de que la salida sea correcta, se continúa en esa dirección. Para que un cambio en una red neuronal se considere inapreciable, la salida debe verse alterada de tal forma que sea indistinguible su efecto del de unos diferentes parámetros iniciales durante el entrenamiento.

De esta forma, una red neuronal dispersa puede contener alrededor de un 80\textasciitilde90\% menos de conexiones entre neuronas, representando esto un ahorro sustancial de energía en términos de acceso a memoria, relevante en sistemas miniaturizados donde cada miliwatt cuenta.

\begin{figure}[h!]
    \centering
    \vspace*{0.5cm}
    \def\svgwidth{0.85\textwidth}
    \input{pdf_tex/sparse_nn/sparse_nn_svgnn.pdf_tex}
    \caption{Red neuronal dispersa Feed-Forward}
    \label{fig:sparse_nn_sample}
\end{figure}

Esta operación de podado puede no solamente limitarse a los pesos que interconectan neuronas, sino que se puede hacer el modelo más eficiente y reducido al podar todos los pesos entrantes en una neurona, eliminando por completo dicha neurona, a la vez sus conexiones posteriores en cadena. El efecto de eliminar una neurona sin conexiones entrantes, que únicamente arroja valores constantes en cada iteración, se puede compensar ajustando el \textit{bias} del receptor que correspondía a esa neurona en las conexiones posteriores.

\subsection{Redes neuronales profundas}
\label{ssec:redes_reuronales_profundas}
El \textit{boom} de la inteligencia artificial en los últimos años viene dado en gran medida por el enorme crecimiento que ha experimentado el sector mediante técnicas de \textit{deep learning}, precisamente en estas redes neuronales profundas o \textit{deep neural networks}.

Y es que dentro de las infinitas posibilidades que nos ofrecen las ``neuronas'' tratadas anteriormente, existe la posibilidad de apilar capa sobre capa, obteniendo a cada capa extra comportamientos y patrones más complejos.
Es precisamente este comportamiento de los sistemas no lineales el que permite reproducir comportamientos exóticos al aumentar el número de neuronas y/o capas (y por tanto la complejidad) del sistema.

Tras sospechas previas de algunos investigadores como George Cybenko acerca de que una red neuronal con al menos una capa oculta es un aproximador universal, y tras demostrar dicho comportamiento para la sigmoide como función de transferencia \cite{cybenko1989approximation}, Kurt Hornik demostró en 1991 que una red neuronal de con al menos una capa oculta es siempre un aproximador universal, independientemente de la función de transferencia, siempre que esta sea no polinomial. A pesar de que una red neuronal de una sola capa oculta pueda ser un aproximador universal, se obtienen aproximaciones mucho más inteligentes y ``económicas'' computacionalmente al contar con más de una capa oculta \cite{hornik1991approximation}.



\section{Redes neuronales densas}
\label{sec:redes_reuronales_densas}
Como se comenta en la sección anterior, independientemente del número de capas de la red, sea profunda o ``convencional'', una red neuronal densa es aquella en la que para $m$ neuronas en la capa $M$, y $n$ neuronas en la capa $N$, donde la neurona $x$ de la capa $M$ se denota $M_{x}$ existe siempre una conexión de $M_{m}$ a $N_{n}$, es decir ($m\longrightarrow n$):

\begin{equation}
\forall m \in M \:\wedge\: \forall n \in N, \: m\longrightarrow n
\label{eq:dense_nn}
\end{equation}

De esta forma, para la ejecución (inferencia) de una red neuronal densa feed-forward, serán necesarias únicamente una simple multiplicación de matrices, una suma de matrices para tener en cuenta los \textit{bias}, y aplicar una función de transferencia.

Esto se puede apreciar en la Figura \ref{fig:nn_matrix_dense}\footnote{Imágenes con este estilo artístico han sido obtenidas de \url{https://ml-cheatsheet.readthedocs.io/en/latest/forwardpropagation.html}}. En ésta se puede ver la inferencia de 4 datos, correspondientes con el número de filas de entrada a las capas. Como también se puede apreciar, la anchura de cada capa se corresponde con el número de columnas de la misma. 

\begin{figure}[h!]
    \centering
    \includegraphics[width=0.85\textwidth]{img/neural_network_matrix_dense.png}
    \caption{Visualización de inferencia en redes neuronales densas}
    \label{fig:nn_matrix_dense}
\end{figure}

De esta forma, si el dato (vectorial por la naturaleza de estas redes) tiene 24 elementos, será necesaria una capa de entrada de 24 neuronas (24 columnas $x_{1} .. x_{24}$ según notación de la Figura \ref{fig:nn_matrix_dense}), y si queremos inferir sobre 1000 de estos datos, esto se puede hacer en una sola ejecución, siendo el número de columnas también de 1000.

O dicho de otra manera, sea la capa $M_{m}$ una capa completamente conexa con $m$ neuronas, a la cual se le introduce un dato $x_{m}$, donde el peso de la neurona $M_{i}$ hacia $N_{j}$ es denotado por $w_{M,i,j}$, y la función de transferencia es $\phi$, tenemos que la salida de una capa será:

\begin{equation}
    x_{N,i} = \phi\left(\sum_{j}w_{N,i,j}x_{M,j}\right)
    \label{eq:dense_nn_eq}
\end{equation}

A esta función se le puede añadir el \textit{bias} de forma explícita ($\sum\left([\dots] x_{M,j}\right) + bias_{N,i}$), o de forma implícita, siendo el \textit{bias} una conexión más, con un valor constante de 1, cuyo peso es el multiplicador que da el \textit{bias} resultante.

\section{Redes neuronales dispersas}
\label{sec:redes_reuronales_dispersas}
Las redes neuronales dispersas cuentan con una ventaja, y es el menor número de conexiones entre neuronas, o directamente el menor número de neuronas en la arquitectura. A pesar de esto, en realidad el fundamento matemático es el mismo, una simple multiplicación de matrices. Sin embargo, a pesar de ser el mismo funcionamiento base, se pueden explotar ciertas características tanto de la estructura de las matrices dispersas como de su multiplicación para obtener mejorías en el tamaño del modelo o en rendimiento, respectivamente.

\subsection{Fundamentos de matrices dispersas}
\label{ssec:fundamentos_matrices_dispersas}
Una matriz es dispersa cuando gran parte de su contenido son únicamente ceros. El cuan dispersa es una matriz se cuantifica mediante el ``grado de dispersión'' o \textit{sparsity}. Una matriz dispersa $10 \times 10$, con tres elementos diferentes de cero (\textit{non-zero elements} o directamente \textit{non-zero(es)}) será una matriz dispersa con una \textit{sparsity} del $97\%$, y una densidad del $3\% \:(=100\%-97\%)$.

\subsection{Almacenamiento de matrices dispersas}
\label{ssec:almacenamiento_matrices_dispersas}
Existen múltiples métodos para el almacenamiento de matrices dispersas en memoria, pero uno de los más sencillos de comprender, y que se usa en la \acrshort{poc} para la creación de las matrices dispersas es el \acrshort{coo} o \textit{\acrlong{coo}}.

Este formato consiste en el almacenado en tres vectores, \texttt{V}, \texttt{C} y \texttt{R}, (\textit{Value}, \textit{Column}, \textit{Row}). Para cada entrada en el vector \texttt{V}, se crea una entrada en la misma posición para los vectores \texttt{C} y \texttt{R}, indicando la columna y fila en la que se ubica el valor \textit{non-zero}. Por diseño, el tamaño de cada uno de estos tres vectores tiene que ser igual del del número de \textit{non-zeroes}, denotado por \texttt{NZ}. Este formato es particularmente cómodo para la creación de matrices dispersas de forma ágil, pero existen formatos más avanzados, tanto en tamaño como en eficiencia, como el \acrshort{csr} o \textit{\acrlong{csr}}\footnote{Más información en \url{https://en.wikipedia.org/wiki/Sparse_matrix\#Compressed_sparse_row_(CSR,\_CRS\_or\_Yale\_format)}}.

De esta forma, a continuación se muestra un ejemplo para una matriz dispersa almacenada en formato \acrshort{coo}. A pesar de que esta matriz no es particularmente dispersa, es únicamente con fines explicativos:

\begin{center}
    $\begin{pmatrix}
        1 & 0 & 0 & 0\\
        0 & 2 & 0 & 0\\
        0 & 3 & 4 & 0\\
        0 & 0 & 0 & 5
    \end{pmatrix}$
    \vspace*{0.5cm}
\begin{lstlisting}[]
NZ = 5

V = [ 1 2 3 4 5 ]
C = [ 0 1 1 2 3 ]
R = [ 0 1 2 2 3 ]
\end{lstlisting}
\end{center}

\subsection{Propiedades del producto de matrices dispersas}
\label{ssec:propiedades_producto_matrices_dispersas}
Al multiplicar dos matrices densas no hay duda de que como resultado se obtendrá una matriz densa, salvo contadas excepciones, como multiplicar una matrix por su inversa. Sabiendo que el producto de una matriz dispersa obtiene el mismo resultado que una multiplicación de matrices densas mediante un algoritmo diferente, podemos clasificar los productos de matrices en función de su \textit{sparsity} de forma genérica.

Esta clasificación, de nuevo, puede variar en función de las propiedades de la matriz, pero es la base sobre la cual se construyen las funciones de \acrshort{blas} (\textit{\acrlong{blas}}) y las implementaciones BLAS\_Sparse. De esta forma, obtendremos los siguientes tipos de producto de matrices:

\begin{itemize}
    \item Densa $\times$ Densa $=$ Densa ($D\times D$)
    \item Dispersa $\times$ Densa $=$ Densa ($d\times D$)
    \item Dispersa $\times$ Dispersa $=$ Dispersa / Densa ($d\times d$)
\end{itemize}

Estos tres tipos principales de productos, en típico \textit{BLAS-fashion} se pueden realizar para tipos de dato \texttt{S}, \texttt{D}, \texttt{C} y \texttt{Z} (\textit{Single}, \textit{Double}, \textit{Single Complex}, \textit{Double Complex}).
De esta forma, podemos obtener las siguientes funciones estándar, a pesar de que en la parte \textit{sparse} de \acrshort{blas} hay menor consenso debido a múltiples librerías que difieren del estándar. Por ejemplo, los tipos especificados en la función se van perdiendo conforme se avanza hacia implementaciones genéricas en C++.

\begin{itemize}
    \item \texttt{[sdcz]gemm()} para $D\times D$
    \item \texttt{[sdcz]usmm()} o \texttt{spmm()} para $d\times D$
    \item \texttt{sp[sdcz]gemm()} o \texttt{spmsp()} para $d\times d$
\end{itemize}

Sabiendo las características de una red neuronal dispersa, y tal como se comenta más adelante en esta memoria, en la \acrshort{poc} se emplea el producto $d\times D$, debido a la naturaleza dispersa de los pesos y a la densa de los datos de entrada. De todas formas, en función de la naturaleza de los datos de entrada podrían emplearse también vectores de entrada dispersos, aunque lo más probable es que para eso quizás una red neuronal feed forward unidimensional no sea lo más apropiado, y ya debamos sugerir otras arquitecturas de red como las convolucionales, debido a que un vector disperso no suele tener mucho sentido, pero una matriz dispersa si.

\subsection{Visualización de red neuronal dispersa}
\label{ssec:visualizacion_nn_dispersa}
Tal como se muestra en la Sección \ref{sec:redes_reuronales_densas}, es sencillo visualizar el proceso de inferencia como una simple multiplicación de matrices, por lo que una multiplicación con una matriz de pesos debería ser sencillo de visualizar de igual manera.

Esto es correcto, es muy sencillo de visualizar de no ser por un pequeño problema que trataré más adelante. Y es que la función de multiplicación de matrices $d\times D$, \texttt{[sdcz]usmm()}, tiene la siguiente firma, que si bien parece algo adecuado para nuestra carga de trabajo tiene un detalle sutil que requiere un poco de detenimiento. A continuación se muestra para el lenguaje C y para simple precisión la firma y parámetros de la función \texttt{BLAS\_susmm}:

\begin{lstlisting}[language=C]
int BLAS_susmm( enum blas_order_type    order,
                enum blas_trans_type    transA,
                int                     nrhs,
                float                   alpha,
                blas_sparse_matrix      A,
                const float *           b,
                int                     ldb,
                float *                 c,
                int                     ldc 
)

/**
 * order    layour of the dense array.
 * transA   Transposition operator for matrix A.
 * nrhs     Number of right hand side columns.
 * A        A valid matrix handle.
 * alpha    Value for $ \alpha $.
 * b        Dense vector b.
 * ldb      Leading dimension of b.
 * c        Dense vector c.
 * ldc      Leading dimension of c.
 */
\end{lstlisting}

El inconveniente de esta función, que calcula $C = \alpha AB + C$ es que la matriz $A$ debe ser dispersa, algo que, fijándonos de nuevo en la Figura \ref{fig:nn_matrix_dense}, no se cumple. Es decir, la matriz dispersa es la de pesos $w$, por lo que parece necesaria una función que trate a $B$ como una matriz dispersa. Dicha función no existe.

\begin{figure}[h!]
    \centering
    \includegraphics[width=\textwidth]{img/neural_network_matrix_sparse/neural_network_matrix_sparse.png}
    \caption{Visualización de inferencia en redes neuronales dispersas}
    \label{fig:nn_matrix_sparse}
\end{figure}

Sin embargo, empleando las propiedades de las matrices, es posible modificar el orden de las mismas, y así poder encajar cada matriz en la firma. Para esto se emplea una propiedad básica de las matrices, $C^{T} = (AB)^{T} = A^{T}B^{T}$. De esta forma, al transponer la matriz de pesos mediante el parámetro \texttt{transA}, y multiplicando esto por la entrada de la red anterior, obtendremos la salida $C^{T}$. Esto supone un pequeño overhead, al tener que transponer la entrada a la capa de entrada, así como la salida de la capa de salida. Sin embargo, dicho overhead puede ser mitigado en gran medida, en función de la arquitectura de la red, así como de la fuente de datos, que es probable que con pequeñas modificaciones pueda exportar los datos transpuestos previamente.

Esta estrategia se puede apreciar en la Figura \ref{fig:nn_matrix_sparse}, donde a diferencia de la Figura \ref{fig:nn_matrix_dense} se han transpuesto los pesos y las entradas, y se ha etiquetado correspondientemente mediante superíndices.

\chapter{Análisis y estado del arte}
\label{chap:analisis_estado_arte}

\lettrine{U}{na} vez aaa.
\chapter{Desarrollo y POC}
\label{chap:desarrollo_poc}

\lettrine{U}{na} vez identificados los puntos de intervención y posible mejora, así como los trabajos ya realizados la propuesta es desarrollar una prueba de concepto, a pesar de que es más que probable que esta primera aproximación básica no mejore el muy optimizado proceso que emplea TensorFlow. El objetivo de esta prueba de concepto es más bien el aprendizaje a bajo nivel, perfilado del código y extracción de características del \textit{workload}, así como intentar proponer mejoras sobre la base ahora tangible. Se ha decidido realizar esto y no una simple medida de rendimiento de redes ya existentes debido a que aquellos son valores medidos infinidad de veces.

Para esta \textit{\acrlong{poc}} se ha creado un \textit{Jupyter Notebook} en el cual se puede ir ejecutando paso a paso cada etapa en la generación y entrenamiento de la red. 

\section{Extracción de valores para modelos TensorFlow}
\label{sec:extraccion_valores_modelo_tf}
El primer paso para la implementación de esta \acrshort{poc} es la extracción de los valores desde el modelo TensorFlow. Este modelo puede venir dado previamente en formato \texttt{.pb} o se puede crear y entrenar desde python sin exportarlo en ningún momento.

Como los modelos más grandes y complejos no suelen ser modelos exclusivamente \textit{feed-forward} con funciones de activación exclusivamente sigmoides, lo más cómodo es simplemente generar una red neuronal de pruebas, entrenarla, y exportar el resultado a C.

\subsection{Desde un modelo guardado}
\label{ssec:desde_modelo_guardado}
Para realizar esto, que queda fuera del objetivo de este trabajo, bastaría con emplear la función \texttt{saved\_model.load} de TensorFlow, sobre un modelo \texttt{.pb} previamente exportado \cite{tensorflow_saved_model}. Esto permitiría importar grandes modelos preentrenados, y realizar mediciones sobre modelos ``del mundo real'', pero implicaría una gran cantidad de horas de trabajo y adaptación, poco adecuadas para la longitud de este proyecto, debido a la gran cantidad de parámetros, tipos de capas y funciones de activación, entre otros, que no se han implementado.

Por esta razón, este proyecto se limita a la extracción para modelos puramente \textit{feed-forward} con función de activación sigmoide (a pesar de que añadir soporte para distintas funciones de activación sería relativamente sencillo de implementar).

\subsection{Desde un modelo de pruebas}
\label{ssec:desde_modelo_de_pruebas}
Este es el camino más sencillo a la hora de implementar lo que recordemos es simplemente un laboratorio básico para poder realizar mediciones y sacar conclusiones.

\subsubsection{Fase común}
\label{sssec:modelo_pruebas_fase_comun}
A continuación se muestran, por orden de aparición en el \textit{notebook}, los pasos de entrenamiento de la red. De aquí en adelante la palabra \acrlong{tf} aparecerá muy habitualmente, por lo que también será referida como simplemente \acrshort{tf}.

\begin{enumerate}
    \item Se importa el conjunto de datos. En este caso, y tras pedir consejo a usuarios más avanzados de \acrshort{tf}, se ha empleado el \textit{dataset} \texttt{german\_credit\_numeric}, debido a la simplicidad de sus entradas y salidas. Este conjunto de datos cuenta con 1000 entradas en el mismo, y se emplea para intentar parametrizar el riesgo crediticio de un individuo mediante redes neuronales.\medskip
\begin{lstlisting}[language=Python]
ds = tfds.load('german_credit_numeric', split='train', as_supervised=True)
\end{lstlisting}

    \item Se indican la función de pérdida, el optimizador y la métrica. Esto, si bien sería un tema muy interesante si este trabajo tratase de obtener buenos resultados con esta red, en este caso es irrelevante que la arquitectura de la red sea buena o no, así como la calidad de su entrenamiento y resultados.

    \item Se crea un modelo secuencial, es decir un modelo ``capa por capa'', en el que cada capa va a continuación de la siguiente. Tras crearlo se \textit{buildea} y se compila.\medskip
\begin{lstlisting}[language=Python]
model = tf.keras.models.Sequential(name= "MySampleModel")
model.add(tf.keras.layers.InputLayer((tamano_entrada,)))
model.add(tf.keras.layers.Dense(units = h0_size, activation="sigmoid"))
model.add(tf.keras.layers.Dense(units = h1_size, activation="sigmoid"))
model.add(tf.keras.layers.Dense(units = 1, activation="sigmoid"))

model.build()

model.compile(loss=fn_perdida, optimizer=optimizador, metrics=metrica)
\end{lstlisting}

    \item Tras la creación del modelo hay que entrenarlo. Este complejo proceso a nivel de implementación, basado en el algoritmo de \textit{backpropagation}, y con todos los detalles que este implica, se puede realizar en, al más puro estilo Python, con un \textit{one-liner}.\medskip
\begin{lstlisting}[language=Python]
history =  model.fit(x=lote_entrenamiento[0], y = lote_entrenamiento[1], batch_size = 20, epochs=num_epochs)
\end{lstlisting}
\end{enumerate}

\subsubsection{Fase de podado}
\label{sssec:modelo_pruebas_fase_podado}
En caso de querer podar la red para aumentar su \textit{sparsity}, el procedimiento es muy similar al descrito en \cite{tensorflow_prune_model}. Los pasos para realizar este procedimiento se describen a continuación:
\begin{enumerate}
    \item Se crean los parámetros para podar el modelo hasta una \textit{sparsity} en este caso del 80\%. Previamente al podado copia el modelo original a un modelo diferente, \texttt{model\_for\_pruning}.\medskip
\begin{lstlisting}[language=Python]
prune_low_magnitude = tfmot.sparsity.keras.prune_low_magnitude

batch_size = 20
sparse_epochs = 2
validation_split = 0.1

end_step = np.ceil(tamano_lote/batch_size).astype(np.int32) * sparse_epochs

pruning_params = {
    'pruning_schedule': tfmot.sparsity.keras.PolynomialDecay(
        initial_sparsity=0,                                                    
        final_sparsity=0.8,
        begin_step=0,
        end_step=end_step
    )
}

# clone the model. If we do not do this, the original model will be altered too
model_for_pruning = tf.keras.models.clone_model(model)

model_for_pruning = prune_low_magnitude(model_for_pruning, **pruning_params)

# `prune_low_magnitude` requires a recompile.
model_for_pruning.compile(optimizer=optimizador, # 'adam',
                loss=fn_perdida,
                metrics=metrica)

model_for_pruning.summary()
\end{lstlisting}
    \item Se entrena el modelo podado. Este procedimiento es muy similar al entrenamiento en una línea explicado previamente, únicamente añadiendo el \textit{callback} necesario para el entrenamiento de un modelo podado.
\begin{lstlisting}[language=Python]
num_epochs =  500

callbacks = [
    tfmot.sparsity.keras.UpdatePruningStep(),
]

history = model_for_pruning.fit(lote_entrenamiento[0], lote_entrenamiento[1], batch_size=batch_size, epochs=num_epochs, validation_split=validation_split, callbacks=callbacks)
history.history
\end{lstlisting}
\end{enumerate}

\subsection{Extracción de valores}
\label{ssec:extraccion_valores}
Ahora que se cuenta con un modelo de red neuronal, de discutible calidad, pero funcionalmente idéntico a otro mejor, con mayor cantidad de neuronas o más capas, se puede comenzar con la extracción de valores de cada capa.

\begin{enumerate}
    \item Primero es necesario obtener tanto los pesos como los \textit{bias}. Para esto se itera sobre las capas del modelo. En cada capa del modelo se puede llamar a la función \texttt{get\_weights()}. En el restultado que devuelve, los pesos se encuentran en un array de arrays en la posición 0, y los \textit{bias} un un array en la 1.\medskip
\begin{lstlisting}[language=Python]
for layer in model.layers:
    print(f"LAYER{idx}_WEIGHTS:")
    print(layer.get_weights()[0])
    print(f"LAYER{idx}_BIAS:")
    print([layer.get_weights()[1]])

# LAYER1_WEIGHTS:
# [[-0.08928536  0.08118167 -1.6078503  -0.29092655 -0.33163825]
#                               ...
#  [-0.16627118 -0.36254594  0.03569769  0.11377215 -0.19968267]]
# LAYER1_BIAS:
# [ 0.00605693  0.00038876  0.29822487  0.01296887 -0.02708104]
# LAYER2_WEIGHTS:
# [[-0.43815556  0.6910728   0.45901924]
#                   ...
#  [ 1.3233485  -0.26594874 -0.5277702 ]]
# LAYER2_BIAS:
# [-0.1541544  -0.04604432  0.15921216]
# LAYER3_WEIGHTS:
# [[-3.4663067 ]
#       ...
#  [ 1.5324626 ]]
# LAYER3_BIAS:
# [1.3300072]
\end{lstlisting}
    Sin embargo, esta aproximación a la hora de obtener los pesos es problemática. Y es que al mostrar los pesos en formato humano (decimal), se pierde precisión. Para solucionar esto se necesita algo más directo y de bajo nivel.
    
    \item La solución a la que se ha llegado es la más evidente, exportar los datos en binario codificado en hexadecimal, para leerlo como dato en C. Realizar esto en Python no es particularmente difícil, y si bien existen mejores formas de interfacear con código C, se ha de recordar que esto simplemente es una prueba de laboratorio y no un producto final.
    
    De esta forma, se define una función para la conversión de datos en hexadecimal:\medskip
\begin{lstlisting}[language=Python]
# https://docs.python.org/3/library/struct.html
FLOAT_BE = ">f"
FLOAT_LE = "<f"
DOUBLE_BE = ">d"
DOUBLE_LE = "<d"

def np_value_to_hex(value, byte_format):
    return bytearray(struct.pack(byte_format, value)).hex()

# byte_format is target format for output
def np_array_to_hex(array, byte_format):
    return map(
        lambda layer: list(
            map(lambda v: np_value_to_hex(v, byte_format), layer)
        ),
        array,
    )
\end{lstlisting}
    Y tras crear dicha función, ahora con un poco de manejo de strings, fácilmente se puede obtener un resultado como el siguiente:\medskip
\begin{lstlisting}[language=Python]
# LAYER1_WEIGHTS:
# bdb6db3e,3da64293,bfcdce0a,be94f453,bea9cc7d
#                     ...
# be2a42fe,beb99f9f,3d1237be,3de90160,be4c799d
# LAYER1_BIAS:
# 3bc67940,39cbd218,3e98b0ee,3c547b67,bcddd911
# LAYER2_WEIGHTS:
# bee055ed,3f30ea26,3eeb0492
#            ...
# 3fa9637c,be882a6f,bf071bf3
# LAYER2_BIAS:
# be1ddaa7,bd3c98f7,3e230883
# LAYER3_WEIGHTS:
# c05dd7f8
#   ...
# 3fc427bc
# LAYER3_BIAS:
# 3faa3dad
\end{lstlisting}

    \item Por último pero no por ello menos importante, se han de exportar también los datos de entrada del propio dataset. Como se discutirá más adelante, el formato del documento que lee el código C compilado contará con una primera línea especificando el número de líneas que va a leer, y un dato por línea. De esta forma, si la entrada contiene 24 parámetros y se ha de inferir 1000 datos, el contenido del documento será de 1001 líneas, siendo la primera \texttt{1000}, y el resto conteniendo los datos a inferir.\medskip
\begin{lstlisting}[language=Python]
# Get original stdout here, in order to print to a file
my_stdout = sys.stdout

with open('input.txt', 'w') as f:
    sys.stdout = f
    # Print dims to beginning of file
    print(len(lote_entrenamiento[0]))
    # and data to the rest of it
    for arr in lote_entrenamiento[0]:
        print(*arr.numpy().tolist(), sep=',')
    # Finally restore stdout
    sys.stdout = my_stdout
\end{lstlisting}
    Al ejecutar el siguiente comando, se crea en el directorio de trabajo el fichero \texttt{input.txt}, que contiene 1000 entradas. El fichero se ha recortado a 5 entradas por conveniencia.\medskip
\begin{lstlisting}
5
3,6,4,13,2,5,1,4,3,28,3,2,2,2,1,1,0,1,0,0,1,0,0,1
4,4,2,6,1,2,2,3,1,23,3,1,2,1,1,0,0,1,0,1,0,0,1,0
4,24,4,20,1,3,2,4,3,37,3,1,1,2,1,1,0,1,0,0,1,0,0,1
4,18,2,11,5,2,2,2,1,21,3,1,1,2,1,0,0,1,0,1,0,0,0,1
4,6,2,13,3,3,1,4,1,62,3,1,1,1,1,0,0,1,0,0,1,0,0,1
\end{lstlisting}

    Conviene recalcar que en este caso el tipo de los datos de entrada es entero. En caso de dicho caso ser flotante, se procedería a exportar en formato hexadecimal para, como previamente, no perder precisión.
\end{enumerate}

\subsubsection{Extracción de datos dispersos}
\label{sssec:extraccion_datos_dispersos}
Para la extracción de datos dispersos se procede de igual forma que con datos densos. Iterando a través de cada una de las capas con funciones como las expuestas previamente, se obtienen resultados como el siguiente:\medskip
\begin{lstlisting}[language=Python]
# LAYER1_WEIGHTS:
# 80000000,80000000,00000000,00000000,00000000
#                     ...
# 3ec9393d,80000000,80000000,00000000,00000000
# LAYER1_BIAS:
# 39fcfbe4,3854f5e8,b82401d6,b8b52b95,ba28adf7
# LAYER2_WEIGHTS:
# 80000000,00000000,80000000
#            ...
# 80000000,80000000,3f5a98cf
# LAYER2_BIAS:
# 3bea3c98,3d964016,3b89974c
# LAYER3_WEIGHTS:
# 80000000
#   ...
# 3f61fe09
# LAYER3_BIAS:
# 3ec81415    
\end{lstlisting}

Como se puede apreciar dentro de las zonas recortadas, la mayor parte de datos son 0x80000000 o 0x00000000, siendo ambos representaciones del número cero en coma flotante. Con ayuda de la función \texttt{nonzero} de \texttt{numpy} se puede extraer facilmente cada valor diferente de cero con sus coordenadas vertical y horizontal para importar más adelante en C, tal como se muestra en la sección siguiente, \nameref{sec:importacion_valores_c}.

\section{Importación de valores en C}
\label{sec:importacion_valores_c}
En esta sección se habla de cómo se importan los datos extraídos del modelo \acrshort{tf} denso o disperso, creado en la sección anterior.

\subsection{Importación de matrices densas}
\label{ssec:importacion_matrices_densas}
El lenguaje de programación C almacena las matrices por filas. Esto es, dada una matriz $M \times N$, se almacenarán en memoria $N$ elemenos de la primera fila, seguidos por $N$ elementos de la segunda, etc. Aprovechando esta característica de C, y haciendo uso del tan granular acceso a bajo nivel que nos permite hacer este lenguaje, se importan las matrices de pesos y \textit{bias} de las diferentes capas desde la zona de memoria constante, o para x86, la sección \texttt{.text}. Esto implica que una vez el programa se carga a memoria no lee ningún archivo salvo la entrada de datos.

Mediante manejo de \textit{strings} en Python, y recordando que la arquitectura x86(y \_64) es \textit{little endian}\footnote{\url{https://es.wikipedia.org/wiki/Endianness}}, los valores constantes se importan al código con las siguientes líneas:\medskip
\begin{lstlisting}[language=C]
#define INPUT_SIZE 24
#define LAYER1_SIZE 5
const fp32 * layer1_weights = (fp32 *)
        "\xf3\x79\x85 /* [...] */ \x8c\xab\xbe";
const fp32 * layer1_bias = (fp32 *)
        "\x3f\x1b\x5f /* [...] */ \x54\x37\x3c";
#define LAYER2_SIZE 3
const fp32 * layer2_weights = (fp32 *)
        "\x16\xbd\xc5 /* [...] */ \x8d\x77\x3f";
const fp32 * layer2_bias = (fp32 *)
        "\xfc\x06\x25 /* [...] */ \xa4\x2a\xbe";
#define LAYER3_SIZE 1
const fp32 * layer3_weights = (fp32 *)
        "\x09\x88\x1d /* [...] */ \x57\x04\x40";
const fp32 * layer3_bias = (fp32 *) "\xc8\x2d\x40\x3d";
\end{lstlisting}

Estos valores, al ser \textit{casteados} a un vector de \texttt{fp32}, más adelante pueden ser accedidos con el \textit{stride} de dicho tipo de datos.

\subsection{Importación de matrices dispersas}
\label{ssec:importacion_matrices_dispersas}
En el caso de las matrices dispersas el almacenamiento de datos no es tan sencillo. Como se comenta en el Capítulo de \nameref{chap:conceptos_basicos}, existen múltiples maneras de representar una matriz dispersa en memoria. La que se emplea en la creación de dicha matriz es la \texttt{COO} (Subsección \ref{ssec:almacenamiento_matrices_dispersas}). De esta manera, es suficiente con exportar una lista de valores acompañado con las coordenadas de cada uno de ellos. Estos valores, tanto los de la matriz como las coordenadas de los mismos, y de igual manera que en la \nameref{ssec:importacion_matrices_densas} se guardan en la sección de texto.

Esto significa que, en comparación con la versión densa, el vector de pesos es mucho más pequeño (en función de la densidad de la matriz que representa), y aparecen dos vectores de coordenadas para \texttt{i} y \texttt{j}:\medskip
\begin{lstlisting}[language=C]
#define INPUT_SIZE 24
#define LAYER1_SIZE 5
#define layer1_nz 24
const fp32 * layer1_weights = (fp32 *)
        "\x7c\xce\xbc /* [...] */ \x39\xc9\x3e";
const int layer1_i[layer1_nz] =
        {1,1,2,3,3,4,8,11,12,12,13,14,15,15,15,17,18,18,19,19,20,21,21,23};
const int layer1_j[layer1_nz] =
        {0,1,1,1,2,0,3,2,1,2,1,4,1,2,4,3,1,2,0,1,2,3,4,0};
const fp32 * layer1_bias = (fp32 *)
        "\xe4\xfb\xfc /* [...] */ \xad\x28\xba";
#define LAYER2_SIZE 3
#define layer2_nz 3
const fp32 * layer2_weights = (fp32 *)
        "\xe2\x81\x50 /* [...] */ \x98\x5a\x3f";
const int layer2_i[layer2_nz] = {2,3,3};
const int layer2_j[layer2_nz] = {2,0,2};
const fp32 * layer2_bias = (fp32 *)
        "\x98\x3c\xea /* [...] */ \x97\x89\x3b";
#define LAYER3_SIZE 1
#define layer3_nz 1
const fp32 * layer3_weights = (fp32 *) "\x09\xfe\x61\x3f";
const int layer3_i[layer3_nz] = {1};
const int layer3_j[layer3_nz] = {0};
const fp32 * layer3_bias = (fp32 *) "\x15\x14\xc8\x3e";
\end{lstlisting}


\section{Generación dinámica de código C a partir del modelo TensorFlow}
\label{sec:generacion_din_modelo_tf}
Tras ver cómo extraer los valores de pesos, \textit{bias}, y datos de entrada del entorno de \acrlong{tf}, se cuenta con todas las herramientas necesarias para generar el código C, o prácticamete cualquier código si se cuenta con las herramientas y conocimientos adecuados.

\subsection{Decisiones previas de diseño}
\label{ssec:decisiones_previas_diseno}
Los útiles comunes a cualquier código generado, sea denso o disperso, se han centralizado en los ficheros \texttt{common.\{c|h\}}.

Primeramente se ha decidido emplear un tipo \texttt{union} para la implementación de los datos en coma flotante, al permitir estos un control superior a nivel de byte, sin complicaciones adicionales como \textit{casting} de punteros.

Los únicos dos tipos de datos implementados son \texttt{fp32} y \texttt{fp64}, con nombres, considero, bastante representativos. El tipo unión, por ejemplo, para el flotante de 32 bits es el siguiente:\medskip
\begin{lstlisting}[language=C]
typedef union _fp32 {
    uint32_t raw;
    float val;
    uint8_t byte[4];
} fp32;
\end{lstlisting}

Así, las funciones implementadas tendrán versiones para flotante de 32 y 64 bits. El sufijo que indica la precisión del flotante será \texttt{\_\_fp32} y \texttt{\_\_fp64} para 32 y 64 bits de precisión, respectivamente. Las pocas funciones implementadas han sido necesarias para el paso de aplicar el \textit{bias}, y siguen esta nomenclatura, a fin de mejorar una futura parametrización extra en la generación del código.

Se ha barajado también la posibilidad de emplear funciones genéricas que brinda el compilador gcc, pero al final por simplicidad, y dada la naturaleza estática de una \textit{\acrlong{poc}}, se ha optado por no incluir estos macros. Un ejemplo para la función \texttt{fast\_sigmoid} sería la siguiente:\medskip
\begin{lstlisting}[language=C]
#define fast_sigmoid(X) _Generic((X), \
fp32:fast_sigmoid__fp32, \
fp64:fast_sigmoid__fp64 \
)(X)

#pragma inline fast_sigmoid__fp32, fast_sigmoid__fp64
fp32 fast_sigmoid__fp32(fp32);
fp64 fast_sigmoid__fp64(fp64);
\end{lstlisting}


\subsection{Matrices densas}
\label{ssec_gdin_matrices_densas}
La forma más sencilla y cronológicamente la primera en ser implementada es la basada en matrices densas. Esta implementación se apoya en la ampliamente disponible librería \texttt{OpenBLAS}\footnote{\url{https://www.openblas.net}}. A pesar de que el proceso de desarrollo, pruebas, automatización, y corrección de posibles bugs pueda ser algo tedioso, en realidad la teoría es moderadamente simple.

Tal como se comenta en el Capítulo de \nameref{chap:conceptos_basicos}, Sección \ref{sec:redes_reuronales_densas}, el programa debe hacer lo siguiente:

\begin{enumerate}
    \item Leer los datos de entrada
    \item Multiplicar los datos de entrada por cada una de las capas
    \item Tratar adecuadamente los datos de salida
\end{enumerate}

Estos sencillos tres pasos tienen su complicación, debido a que hay que implementar todo de cero y automatizarlo.

\begin{enumerate}
    \item Leer los datos de entrada es quizás de lo más sencillo. Es manejo de ficheros básico mediante \texttt{fscanf}:\medskip
\begin{lstlisting}[language=C]
FILE *inputfile;
fprintf(stderr, "*** Opening %s as input ***\n", argv[1]);
if((inputfile = fopen(argv[1], "r")) == NULL){
    fprintf(stderr, "  -> Error: file %s does not exist\n", argv[1]);
    exit(-1);
}

// Parse input dims and allocate memory consquently
int input_dim;
fscanf(inputfile, "%d", &input_dim);

fp32 * input;
input = malloc(input_dim * INPUT_SIZE * sizeof(fp32));

// Read input
for(int i = 0; i < INPUT_SIZE * input_dim; i++){
    fscanf(inputfile, "%f,", &input[i].val);
}

// Finish reading input so let's close it
fclose(inputfile);
\end{lstlisting}

    \item Para realizar la multiplicación de matrices, y siguiendo con este ejemplo, se emplea la función \texttt{cblas\_sgemm} para la mutiplicación. El resultado se guarda de forma aditiva sobre \texttt{C}, por lo que es necesario que dicha matriz esté a cero.\medskip
\begin{lstlisting}[language=C]
fprintf(stderr, "*** SGEMM %s x %s ***\n", "layer0", "layer1");
cblas_sgemm(CblasRowMajor, CblasNoTrans, CblasNoTrans, input_dim,
            LAYER1_SIZE, LAYER0_SIZE, 1.f, (float *) layer0_out,
            LAYER0_SIZE, (float *) layer1_weights, LAYER1_SIZE,
            1.f, (float *) layer1_out, LAYER1_SIZE);
\end{lstlisting}

    Tras la multiplicación de \texttt{A} por \texttt{B}, aditiva sobre \texttt{C}, se han de aplicar los \textit{bias} y la función de transferencia. Si bien podría haberse sumado sobre los \textit{bias} en la función previa, es mejor realizar esta operación en conjunción con la aplicación de la función de transferencia. Esto es así ya que, especialmente si la función se paraleliza, probablemente \texttt{map\_and\_bias} sea \textit{memory-bound}, y un poco de computación extra no vaya a marcar la diferencia. Por el contrario, si se realizara operación sobre una matriz pre inicializada, se necesitaría replicar los valores del \textit{bias} \texttt{input\_dim} veces, lo cual también es \textit{memory-bound}.
\begin{lstlisting}[language=C]
fprintf(stderr, "*** Map and Bias %s with %s function ***\n", "layer1_out", "sigmoid__fp32");
map_and_bias__fp32(layer1_out, layer1_bias, input_dim, LAYER1_SIZE, 'N', sigmoid__fp32);
\end{lstlisting}

    La implementación inicial de \texttt{map\_and\_bias} es similar a la siguiente:
\begin{lstlisting}[language=C]
void map_and_bias__fp32(fp32 *restrict A, const fp32 *restrict bias, const uint32_t M, const uint32_t N, fp32 (* map_function)(fp32 x)){
    for (uint32_t i = 0; i < M; i++){
        for(uint32_t j = 0; j < N; j++){
            A[i*N+j].val = map_function((fp32)(A[i*N+j].val +
                                        bias[j].val)).val;
        }
    }
}
\end{lstlisting}

    \item Tras repetir el paso anterior tantas veces como capas tenga la arquitectura, los resultados se pueden encontrar en la matriz \texttt{layerN\_out}, siendo \texttt{N} el número de capas. Estos datos pueden ser exportados, mostrados, contados, etc. Como en este caso interesa perfilar y visualizar las características del \texttt{workload}, lo más lógico es primero probar con una cantidad de reducida de datos para verificar el correcto funcionamiento del código generado.
    
    Tras verificar el correcto funcionamiento, a pesar de que no sería necesario nada más, se añade un pequeño resumen, que clasifica el número de predicciones. Esto es realmente útil para comparar con la cifra que arroja \acrshort{tf}.
\end{enumerate}


\subsection{Matrices dispersas}
\label{ssec_gdin_matrices_dispersas}
Una vez comprendido cómo se procede con matrices de pesos densas, y apoyado por la Sección \ref{sec:redes_reuronales_dispersas} de los \nameref{chap:conceptos_basicos}, los pasos a mayores que se deben realizar son:

\begin{enumerate}
    \item Transponer la entrada y salida.
    \item Ajustar los parámetros de las funciones densas.
    \item Sustituir las matrices de pesos dispersas por densas.
    \item Sustituir las funciones densas por sus equivalentes dispersas.
\end{enumerate}

Para realizar estos pasos primero se necesita una librería de \acrshort{blas} que soporte matrices dispersas, y exponga una cabecera para C. A ser posible, y como en cualquier proyecto, es conveniente que dicha dependencia sea moderadamente conocida, dado que eso asegurará una mayor calidad del código interno, así como un mejor soporte. Cumpliendo estos requerimientos, se ha optado por la utilización de \texttt{librsb}\footnote{Página principal \url{http://librsb.sourceforge.net}, con paquete en el AUR (ArchLinux User Repository) \url{https://aur.archlinux.org/packages/librsb}, así como parquete mantenido en Ubuntu \url{https://packages.ubuntu.com/jammy/librsb-dev}}, de la cual se emplea su \textit{Sparse BLAS interface}.

De esta forma, los pasos para adaptar el código para trabajar con matrices dispersas en los pesos son:

\begin{enumerate}
    \item Para transponer las matrices de entrada y de salida se emplea una función \textit{built-in} de OpenBLAS, por lo que en este aspecto el código está ligado al dicha librería. De todos modos, no parece una dependencia muy complicada de sobrellevar, puesto que esta implementación es muy habitual en la industria.\medskip
\begin{lstlisting}[language=C]
fprintf(stderr, "*** Transposing input out-of-place ***\n");
cblas_somatcopy(CblasRowMajor, CblasTrans, input_dim, LAYER0_SIZE, 1.f, (float *) input, LAYER0_SIZE, (float *) temp_input, input_dim);
\end{lstlisting}

    Este procedimiento, que se realiza \textit{out-of-place} por la mejoría en rendimiento que supone al transponer matrices no cuadradas, cuenta con un procedimiento previo y posterior con punteros en el que no es necesario profundizar, pero si que se ha omitido un pequeño detalle en el anterior campo de código. Y es que debido a la forma en la que C almacena las matrices en memoria, cuando la matriz es un vector ``vertical'' (una matriz con varias filas y una sola columna), en memoria dicho vector se representa de igual manera que si el vector es ``horizontal'' (una matriz con una sola fila y varias columnas). Por esta razón, y debido a que los tamaños de cada capa se conocen en tiempo de compilación, es posible ahorrar un pequeño tiempo y espacio de código al añadir un simple macro de preprocesador. A continuación se muestra el bloque que realiza la transposición de la capa de salida de forma transparente al resto del programa:\medskip
\begin{lstlisting}[language=C]
#if LAYER3_SIZE != 1
fp32 * temp_output = malloc(input_dim * LAYER3_SIZE * sizeof(fp32));
fprintf(stderr, "*** Transposing output out-of-place ***\n");
cblas_somatcopy(CblasRowMajor, CblasTrans, LAYER3_SIZE, input_dim, 1.f, (float *) layer3_out, input_dim, (float *) temp_output, LAYER3_SIZE);
free(layer3_out);
layer3_out = temp_output;
temp_output = NULL;
#endif
\end{lstlisting}

    \item En cuanto a ``ajustar los parámetros de las funciones densas'', este es un paso intermedo que no se aprecia en la versión final. En este paso se comprueba que el resultado no varía al realizar la transformación necesaria para funcionar con las funciones \textit{sparse} de \acrshort{blas}. Fuera de los cambios en los parámetros, se hace necesario un cambio en la función \texttt{map\_and\_bias}, que ahora debe aplicar los \textit{bias} en un \textit{layout} de memoria diferente. Para esto se añade un parámetro \texttt{transA}, quedando la modificación en:\medskip
\begin{lstlisting}[language=C]
void map_and_bias__fp32(fp32 *restrict A, const fp32 *restrict bias, const uint32_t M, const uint32_t N, const char transA, fp32 (* map_function)(fp32 x)){
    switch (transA){
    case 'N':
    case 'n':
        // [...] algoritmo inicial
        break;

    case 'T':
    case 't':
        for (uint32_t i = 0; i < M; i++){
            for(uint32_t j = 0; j < N; j++){
                A[i*N+j].val = map_function((fp32)(A[i*N+j].val 
                                            + bias[i].val)).val;
            }                                 // ^^^^^
        }                    // aquí se cambia el orden de acceso a bias
        break;
    }
}
\end{lstlisting}

    \item El siguiente paso es sustituir las matrices densas por matrices dispersas. Para esto se emplea el mismo método descrito anteriormente de \textit{hardcodeo} de datos en la Subsección \ref{ssec:importacion_matrices_dispersas}. De esta forma, la previamente mencionada \texttt{layer1\_weights} se divide en el vector de valores y los vectores de coordenadas.
    
    Estos datos son necesarios para la creación de las matrices dispersas, indicando \texttt{layerN\_i} y \texttt{layerN\_j} la fila y la columna de cada valor en el array de pesos, respectivamente. Además, se emplea el número de \textit{nonzeroes} para no sólo la creación de la matriz en la función, sino además como comprobación extra en tiempo de compilación. De esta forma, las matrices dispersas se crean de la siguiente manera:
\begin{lstlisting}[language=C]
blas_sparse_matrix layer1_sp_weights = blas_invalid_handle;
/* [...] */

layer1_sp_weights = BLAS_suscr_begin(LAYER0_SIZE, LAYER1_SIZE);
BLAS_suscr_insert_entries(layer1_sp_weights, layer1_nz, (float *)
                          layer1_weights, layer1_i, layer1_j);
BLAS_suscr_end(layer1_sp_weights);
\end{lstlisting}

    \item Finalmente queda emplear las funciones adecuadas para la mutiplicaciones de matrices $d \times D$. Este proceso consiste básicamente en sustituir la función \texttt{gemm} por una función \texttt{usmm}, así como indicar a \texttt{map\_and\_bias} que la matriz de resultados se encuentra transpuesta. Los valores de $M$ y $N$ (siendo la matriz \texttt{C} de dimensiones $M\times N$) también deben intercambiarse.
\begin{lstlisting}[language=C]
fprintf(stderr, "*** SUSMM %s(T) x %s(T) ***\n", "layer3", "layer2");
BLAS_susmm(blas_rowmajor, blas_trans, input_dim, 1.f, layer3_sp_weights,
           (float *) layer2_out, input_dim, (float *) layer3_out,
           input_dim);
fprintf(stderr, "*** Map and Bias %s(T) with %s function ***\n",
                "layer3_out", "sigmoid__fp32");
map_and_bias__fp32(layer3_out, layer3_bias, LAYER3_SIZE, input_dim,
                   'T', sigmoid__fp32);
\end{lstlisting}
\end{enumerate}

\section{Posibilidades de esta aproximación}
\label{sec:posibilidades_esta_aproximacion}
Las posibilidades de esta aproximación son muchas. Generar código C a partir de un modelo que precisa de soporte en Python es algo especialmente útil para sistemas sin dicho soporte. Es cierto que existe TensorFlow Lite (TFLite) para sistemas embebidos, pero la flexibilidad que aporta crear un código C, la posibilidad de paralelizar tanto en memoria compartida como distribuida dicho código, así como la experiencia obtenida en la realización del proyecto son motivos de peso.

En conjunción con todo esto, el hecho de emplear matrices dispersas viene no solamente condicionado por la literatura ya existente, que destaca los beneficios en cuanto a la proporción rendimiento/precisión de redes podadas, sino que viene también motivada por las labores de investigación en cuanto a análisis estático de código y empaquetamiento de operandos para vectorización conducidas en \cite{exploring_simd_instructions_packing_marcos_horro}.

A continuación se mencionan según la subsección algunas de las posibilidades que abre este concepto.

\subsection{En plataformas de cómputo generalistas}
\label{ssec:posibilidades_en_computo_generalistas}
En una plataforma de cómputo generalista, el mundo de tanto el entrenamiento como la inferencia está dominado por las \acrshort{gpu} (\textit{\acrlong{gpu}}). Estos dispositivos son muy convenientes para las cargas de trabajo que se manejan en deep learning, y aportan aceleraciones astronómicas sobre su contraparte en CPU. A pesar de que librerías como \texttt{numpy} están altamente optimizadas para su ejecución en multitud de redes, incluyendo las basadas en pura multiplicación de matrices dispersas, es cierto que incorporando la investigación conducida por Marcos Horro en \cite{marta_marcos_horro_9804589}, así como otras posibles mejoras posteriores, es posible que se mejore el rendimiento o consumo energético de una red basada en matrices dispersas.

Por esta razón, a pesar de que a día de hoy es difícil mejorar el rendimiento de una red dispersa sin afectar a su precisión, incorporando estas nuevas técnicas se podría obtener una mayor precisión con igual rendimiento, o un mayor rendimiento a igual precisión. Además, esta inferencia podría realizarse en entornos de memoria compartida o distribuida, únicamente alterando la correspondiente zona generación de código para incorporar los cambios necesarios.

\subsection{En plataformas de cómputo heterogéneas}
\label{ssec:posibilidades_en_computo_heterogeneas}
Siguiendo la línea de la Subsección anterior, dichas implementaciones tanto en memoria compartida como distribuida podrían hacer uso de diversos aceleradores. Ante esta posibilidad, aparecen múltiples tipos de aceleradores válidos para este cometido.

\subsubsection{GPU}
\label{sssec:heterogeneas_gpu}
Como ya se comenta levemente en la Subsección \ref{ssec:posibilidades_en_computo_generalistas}, las GPUs son piezas de hardware excelentes para la inferencia, por lo que contar con una mayor cantidad de ellas en un entorno escalable de memoria distribuida es algo deseable.

Generar código para estos dispositivos, si bien cuenta con sus detalles, sería posible empleando un \textit{backend} para CUDA u OpenCL y similares.

\subsubsection{FPGA}
\label{sssec:heterogeneas_fpga}
Las \acrshort{fpga} o \textit{\acrlong{fpga}} son dispositivos programables a nivel de hardware. Si bien es conveniente no entrar en detalles de implementación, la posibilidad de cambiar las rutas por donde fluyen los datos, así como de interconectar diferentes piezas de hardware como memorias, \acrshort{cpu} y más recursos hardware según sea más conveniente, permite un aprovechamiento sin igual de su arquitectura, mediante por ejemplo la creación de \textit{pipelines}, etc.

Si bien realizar esto no es trivial, sabiendo que es posible reducir las operaciones con matrices dispersas a una secuencia finita de operaciones \textit{hardcodeadas}, no resulta descabellado pensar en la opción de implementar dicho \textit{backend} para \acrshort{fpga} mediante una arquitectura adecuada.

\subsubsection{ASIC}
\label{sssec:heterogeneas_asic}
El siguiente paso en el aumento de rendimiento (absoluto o por unidad de energía) es recurrir a circuitos de aplicación específica, \acrshort{asic} (\textit{\acrlong{asic}}). Con este término \acrshort{asic} se engloba no solo a los chips diseñados de cero con una finalidad, sino también a los procesadores de aplicación específica o \acrshort{asip} (\textit{\acrlong{asip}}). Estos circuitos programados en una \acrshort{isa} (\textit{\acrlong{isa}}), o directamente diseñados en un lenguaje de diseño hardware como VHDL o Verilog permiten ese extra de eficiencia necesario en sistemas embebidos o de muy alta replicación y por tanto consumo agregado.

Como se puede imaginar, si bien difícil, no es imposible implementar un backend que exporte modelos \acrlong{tf} a VHDL o Verilog para los \acrshort{asic}, y tampoco debería ser difícil compilar código C para un \acrshort{asip}. Las optimizaciones sin embargo deberían venir de cómo se implementa el producto de matrices y optras operaciones relevantes como la función de transferencia.

Por último, en estos sistemas embebidos podría ser posible emplear otras técnicas de bajo nivel, imposibles de ejecutar en un ordenador convencional debido a la naturaleza de un sistema operativo, tales como permitir cierta tasa de error en los accesos a memoria a cambio de una reducción en el voltaje operativo de la misma (Sección \ref{ssec:otras_optimizaciones}).
\chapter{Medida del rendimiento y perfilado}
\label{chap:medida_rendimiento_perfilado}

\lettrine{T}{ras} plantear la problemática y construir la prueba de concepto, en este capítulo se muestran los resultados de las medidas de rendimiento tanto para los códigos densos como dispersos. Estos resultados se comparan con los que se pueden ver en la literatura ya existente, para comprobar si la red implementada se adhiere a los patrones indicados por ejemplo en la Sección \ref{sec:investigacion_optimizaciones_propuestas}.

\section{Metodología}
\label{sec:metodologia}
La medida del rendimiento de las pruebas de concepto no es tarea trivial. Al ser una simple Prueba de Concpto la función \texttt{map\_and\_bias} no está debidamente optimizada con OpenMP. Realizar esta optimización no es difícil, pero quizás sería innecesario cuando lo que se pretende es medir las posibilidades de una aproximación, y no optimizar y trabajar a fondo en ella.

Por esta razón se ha decidido simplemente implementar de forma sencilla la generación de código con las librerías OpenBLAS y librsb, sin añadir OpenMP u optimizaciones mayores en funciones auxiliares. El \textit{overhead} que añade el tratamiento auxiliar de datos es lo suficientemente bajo como para no necesitar paralelización en un entorno de pruebas.

\subsection{Común}
\label{ssec:comun_metodologia}
Para la medida del rendimiento y perfilado se generan dos redes neuronales de un tamaño absurdamente grande, una densa con capas completamente conexas con un anchos tal que \{capa de entrada, $n\:\times\:$capas ocultas, capa de salida\} = \{24, 500, 800, 1000, 1200, 600, 400, 200, 100, 50, 1\}, y otra dispersa con una \textit{sparsity} del 95\% generada a partir del modelo denso. Evidentemente, para el problema que se pretende resolver, estas redes están completamente sobredimensionadas y son cuanto menos inútiles, puesto que debido a su enorme tamaño lo único que aprenden durante el proceso de entrenamiento es a marcar como positivos todos los \textit{inputs}.

Esto, que para un ingeniero en inteligencia artificial sería un enorme fracaso, en este ámbito es algo completamente indiferente, ya que a pesar de la inutilidad de la red creada, esta sigue realizando las cargas de trabajo típicas de una red neuronal adecuada, esto es, multiplica matrices, suma los \textit{bias} y aplica funciones de transferencia.

En ambas ejecuciones se emplean los mismos datos de entrada, que consisten en el fichero \texttt{inupt.txt} generado con la función tratada en el Punto 3 de la Sección \ref{ssec:extraccion_valores}, replicado 100 veces, para obtener $1000 \times 100 = 100000$ datos de entrada.

\subsection{Medida del rendimiento}
\label{ssec:medida_rendimiento_metodologia}
Para la medida del rendimiento se emplea el programa \texttt{hyperfine}\footnote{\url{https://github.com/sharkdp/hyperfine}} y compilaciones de \textit{release} (opción \texttt{-s} o \textit{strip}). Mediante esta herramienta se realizan 15 ejecuciones de las cuales se calcula la media y desviación típica automáticamente, con 3 ejecuciones previas de calentamiento. Siendo un ejemplo de binario generado \texttt{dense.out}, la ejecución del \textit{benchmark} sería tal que:\medskip
\begin{lstlisting}[language=bash]
hyperfine --warmup 3 --runs 15 './dense.out input.txt'
\end{lstlisting}

Tanto las ejecuciones de medición de tiempos como los perfilados se realizan en un equipo Xiaomi Mi Notebook 15 con las características visibles en la Tabla \ref{tb:especificaciones_xiaomi}.
\begin{table}
\centering
\begin{tabular}{|c|c|}
    \hline
    CPU & 11th Gen Intel(R) Core(TM) i7-11370H @ 3.30GHz\\\hline
    RAM & 16 GB @ 3200MHz (Dual Channel)\\\hline
    Sistema & Ubuntu 20.04 - Linux TODO COMPLETAR\\\hline
    CC & gcc 11 TODO CHECKEAR VERSIÓN\\\hline
    OpenBLAS & libopenblas-dev focal, v0.3.8\\\hline
    librsb & librsb-dev focal, v1.2.0.8\\\hline
    oneapi & Intel oneAPI 2022\\\hline
\end{tabular}
\caption{\label{tb:especificaciones_xiaomi}Especificaciones técnicas del equipo de pruebas}
\end{table}


\subsection{Análisis y perfilado}
\label{ssec:analisis_perfilado_metodologia}
Para el análisis y perfilado se emplea el programa Intel Advisor, el cual permite realizar modelos \textit{roofline} de partes individualizadas de programas, incluyendo las funciones de su librería \acrshort{mkl} (\textit{\acrlong{mkl}}). Esto es especialmente útil para perfilar la implementación densa, que emplea funciones de \texttt{cblas} ampliamente utilizadas.

Sin embargo, esta universalidad se pierde con la versión dispersa (\textit{sparse}), ya que se emplean funciones \textit{built-in} de OpenBLAS, así como funciones propias de librsb, lo que hace que cambiar a la \acrshort{mkl} requiera una reprogramación de ciertas líneas del código para poder perfilar las funciones de librería con Intel Advisor. Esto, que inicialmente puede parecer un problema, no lo es tanto si se razona con respecto al \textit{roofline} de la versión densa.

\subsection{Compilación}
\label{ssec:compilacion_metodologia}
Para compilar las versiones densa y dispersa se requiere intercambiar OpenBLAS por la Intel \acrshort{mkl}, así como desactivar el \textit{stripping} del binario para activar los símbolos de depuración (susituir parámetro \texttt{-s} por \texttt{-g}). Esto implica modificar las líneas de compilación genéricas que se pueden encontrar en el fichero \texttt{.ipynb}, tal como se muestra a continuación.

\subsubsection{Código denso}
Para la obtención del \textit{roofline model} de la carga de trabajo, es necesario o bien calcularlo manualmente, o bien emplear alguna herramienta adecuada para ello. Como ya se comenta previamente, se emplea Intel Advisor para el perfilado del código, por lo que es necesario compilar el código denso con una configuración que sustituya OpenBLAS por \acrshort{mkl}. Para esto se puede compilar de las siguientes formas:\medskip
\begin{lstlisting}[language=bash]
# Para una compilación convencional sin depuración con OpenBLAS, sería necesario únicamente ejecutar
gcc -march=native -O3 -s *.c -o dense.out -lm -lcblas       # En Arch Linux
gcc -march=native -O3 -s *.c -o dense.out -lm -lopenblas    # En Ubuntu

# Sin embargo, con propósitos de perfilado con Intel Advisor, en un entorno bash donde se haya realizado `source /opt/intel/oneapi/setvar.sh` se ha de compilar con:
gcc -march=native -O3 -g3 -DMKL_ILP64 -m64 -I"${MKLROOT}/include" *.c -o dense.out -L${MKLROOT}/lib/intel64 -Wl,--no-as-needed -lmkl_intel_ilp64 -lmkl_gnu_thread -lmkl_core -lgomp -lpthread -lm -ldl
\end{lstlisting}

\subsubsection{Código \textit{sparse}}
En este caso, debido al uso de \texttt{librsb} como librería de Sparse BLAS, la herramienta de perfilado y análisis de código Intel Advisor, a pesar de recompilar la librería con \textit{flags} de \textit{debug}, no es capaz de analizar el código de librería. Una adaptación a la librería Intel MKL, a pesar de no ser imposible, no es conveniente. Por esto mismo más adelante en este capítulo se estima el rendimiento y posibilidades de mejora del código disperso, en función a los resultados con respecto al denso. Para compilar el código para \textit{release}, las líneas de compilación son las siguientes:\medskip
\begin{lstlisting}[language=bash]
# Para una compilación convencional sin depuración con OpenBLAS, sería necesario únicamente ejecutar
gcc -march=native -O3 -s *.c -o sparse.out -lm -lrsb -lcblas # En Arch Linux
gcc -march=native -O3 -s *.c -o sparse.out -lm -lrsb -lopenblas  # En Ubuntu
\end{lstlisting}

\section{Medida de rendimiento, perfilado y \textit{roofline model}}
\label{sec:medida_perfilado_roofline}
En esta sección se muestran los resultados obtenidos según la metodología descrita en la sección anterior, así como se razonan los posibles resultados que no se han podido obtener debido a limitaciones en el análisis.

\subsection{Código denso}
Los resultados obtenidos para la red neuronal densa son los siguientes:

\begin{center}
Tiempo $(\overline{t} \pm \sigma)$ = 10,300 s $\pm$ 0,152 s

Rangos (min \ldots\ max) = 10,022 s \ldots\ 10,568 s
\end{center}

Estos resultados se obtienen además realizando un excelente uso de la memoria caché, tal como se muestra en el modelo \textit{roofline} en la Figura \ref{fig:roofline_dense}.

\begin{figure}[h!]
    \centering
    \includegraphics[width=\textwidth]{img/roofline_dense.png}
    \caption{\textit{Roofline model} del código denso}
    \label{fig:roofline_dense}
\end{figure}

Como se puede apreciar en el modelo, el \textit{workload} de interés, que es el que se puede encontrar en la parte superior derecha (coloreado en rojo y verde), corresponde a las funciones \texttt{cblas\_sgemm}. Estas funciones están fuertemente optimizadas y emplean instrucciones AVX512, como se puede observar en los detalles de la carga (Figura \ref{fig:roofline_dense_details}).

Fijándose con atención se pueden ver funciones con un considerablemente menor desempeño en la parte inferior izquierda. Estos puntos corresponden a \texttt{map\_and\_bias} y sucesivas llamadas a otras funciones como \texttt{expf}. Tal como se comenta previamente, una paralelización es sencilla de funciones auxiliares es sencilla. Sin embargo y tal como se puede ver en el modelo, se pueden distinguir perfectamente los componentes de dichas funciones, y siendo el tiempo de ejecución constante para dos redes con las mismas dimensiones, es sencillo discernir qué mejorías vienen causadas por un producto de matrices más eficiente.    

\begin{figure}[h!]
    \centering
    \includegraphics[width=0.5\textwidth]{img/roofline_dense_details.png}
    \caption{Detalles de \texttt{cblas\_sgemm} en el modelo del código denso}
    \label{fig:roofline_dense_details}
\end{figure}

\subsection{Código \textit{sparse}}
Por otro lado, los resultados obtenidos para la red neuronal dispersa son los siguientes:

\begin{center}
Tiempo $(\overline{t} \pm \sigma)$ = 13,862 s $\pm$ 0,710 s

Rangos (min \ldots\ max) = 12,635 s \ldots\ 15,285 s
\end{center}

En este caso, debido al empleo de la librería librsb, el modelo \textit{roofline} no contiene información de utilidad (Figura \ref{fig:roofline_sparse_details}). Esto, que a todas luces es un problema, deja de serlo si se realiza un sencillo razonamiento.

\begin{figure}[h!]
    \centering
    \includegraphics[width=\textwidth]{img/roofline_sparse.png}
    \caption{\textit{Roofline model} del código disperso}
    \label{fig:roofline_sparse_details}
\end{figure}

Teniendo en cuenta que los tiempos de ejecución de una red dispersa al 95\% son \textasciitilde25\% superiores a los de la red completamente conexa, aún teniendo que procesar un 95\% menos de datos, se puede realizar una estimación del número teórico de FLOP necesarios para las operaciones de multiplicación.

\subsubsection{Aproximación teórica}
Sean dos matrices $d \times D$, de dimensiones $m \times n$ y $n \times k$ y la matriz dispersa con una densidad del 50\%, siendo $\#nz$ el número de valores no-cero en la misma:

\begin{gather}
    \begin{pmatrix}
        0 & d_{12}\\
        d_{21} & 0\\
        0 & d_{32}
    \end{pmatrix}	
    \begin{pmatrix}
        D_{11} & \dots\\
        D_{21} & \dots
    \end{pmatrix}
    =
    \begin{pmatrix}
        C_{11} & \dots\\
        C_{21} & \dots\\
        C_{31} & \dots
    \end{pmatrix}
    \label{eq:flops_sparsity_eq}
\end{gather}

Comenzando con $C$ en una región de memoria a ceros, la secuencia de operaciones a realizar sería la siguiente para este ejemplo:

\begin{center}
    $C_{11} \pluseq d_{12} \cdot D_{21}$\\
    $C_{21} \pluseq d_{21} \cdot D_{11}$\\
    $C_{31} \pluseq d_{32} \cdot D_{21}$\\
\end{center}

Esta secuencia \textit{hardcodeada} se repetirá $k$ veces, (el número columnas de $D$ y $C$). Teniendo en cuenta que una operación FMA realiza dos FLOP, se puede concluir que el número de FLOP necesarios para el cálculo de la matriz resultado $C$ será $2 \cdot \#nz \cdot k$.

\subsubsection{Estimación del rendimiento}
Teniendo en cuenta que las matrices dispersas de pesos en el código disperso tienen una densidad del 5\% (o lo que es lo mismo, una \textit{sparsity} del 95\%), desde un punto de vista teórico se puede concluír que se deberían realizar un 95\% menos de operaciones en punto flotante.

Sin embargo esta reducción en el número de operaciones necesarias, debido a múltiples factores como el principio de localidad, estructura interna a la hora de almacenar la matriz, estructuras de control, así como optimizaciones internas de la propia librería, no se corresponde con una disminución del tiempo de ejecución, sino más bien todo lo contrario. Esta reducción en el número de FLOPS no lo es sin embargo en intensidad aritmética, puesto que también se cuenta con menor número de bytes de datos, resultando en un hipotético descenso en el eje $y$ en el modelo.

Por esta razón, el modelo \textit{roofline} resultante, que el programa Intel Advisor es incapaz de generar, se vería similar a lo que se puede observar en la Figura \ref{fig:roofline_sparse_estimado}.

\begin{figure}[h!]
    \centering
    \includegraphics[width=\textwidth]{img/roofline_sparse_estimado.png}
    \caption{Estimación del \textit{roofline model} del código disperso}
    \label{fig:roofline_sparse_estimado}
\end{figure}

Estos decepcionantes resultados, sin embargo, no son tan desalentadores, ya que implica que existe un amplio margen de mejora mediante análisis estático de código y generación de instrucciones FMA y de control \textit{ad-hoc}, que incluso pueden ser vectorizadas mediante el empleo de la herramienta MARTA\footnote{\url{https://github.com/UDC-GAC/MARTA}}.
% \include{contido/aplicacion_web}
% \include{contido/planificacion_costes}
% \chapter{Conclusiones}
\label{chap:conclusiones}

\lettrine{E}{n} este último capítulo se realiza un balance general de este \acrlong{tfm}, se comenta brevemente la relación con la titulación, y se tratan las líneas de investigación futuras.

\section{Conclusiones}
El objetivo principal de este trabajo era el análisis, así como una posible mejora de los tiempos de ejecución de modelos de inteligencia artificial en fase de inferencia, así como de su consumo energético. Estos objetivos se han cumplido satisfactoriamente. No solamente se ha analizado en detalle el producto de matrices para inteligencia artificial, siendo este código una porción muy relevante del tiempo de ejecución de múltiples arquitecturas de redes neuronales, sino que se ha implementado de cero y en código C una red neuronal, sobre la que se han propuesto mejoras útiles bajo ciertas condiciones de \textit{sparsity}.

Los resultados obtenidos son buenos y esperanzadores, ya que mediante una paralelización sencilla, ausencia de vectorización y una ordenación de datos no necesariamente optimizada, se obtienen, a partir de aproximadamente el 87,5\% de \textit{sparsity}, tiempos de ejecución crecientemente mejores con respecto a los obtenidos por librerías \acrshort{blas} estándar en la industria. Este nivel de \textit{sparsity} es algo elevado para una red de propósito general, por lo que las optimizaciones propuestas, de momento, no pueden ser empleadas en la resolución de cualquier problema con un índice de dispersión más modesto. Sin embargo, tal como se puede apreciar en la Figura \ref{fig:grafica_sparse_vs_dense} (Subsección \ref{ssec:podado_y_redes_dispersas}), para una red correctamente diseñada y entrenada, es en los valores intermedios entre 80\% y 90\% de \textit{sparsity} donde se obtiene la mejor relación entre rendimiento y precisión.

Por último, y siendo este un objetivo secundario pero no por ello menos importantes, y si bien la medida del consumo energético no se ha realizado detalladamente, resulta evidente que, tal como se comenta en capítulos anteriores, como por ejemplo en la Subsección \ref{ssec:xpu}, una menor cantidad de transferencias desde memoria principal, así como una menor cantidad de operaciones en CPU, siempre que el diseño de la microarquitectura acompañe, debe necesariamente traducirse en una menor cantidad de energía consumida.

\section{Relación con la titulación}
En este trabajo se han ampliado extensivamente herramientas de depuración y perfilado, tratadas principalmente en la asignatura de Herramientas para HPC, así como una (simple) paralelización con \texttt{OpenMP} del código \textit{point-to-point}. Esta disciplina, si bien no se ha necesitado un nivel de maestría elevado con el framework en cuestión, ha sido adquirida en las asignaturas de Programación Paralela y Programación Paralela Avanzada.

Por otro lado, el hecho de poder plantear ciertos razonamientos con respecto a la correcta utilización de la jerarquía de memoria, así como la implementación de un algoritmo optimizado para ello, no sería posible sin el trabajo realizado durante los últimos años tanto en el Grado en Ingeniería Informática como en este Máster en Computación de Altas Prestaciones.

\section{Trabajo futuro}
Las líneas de trabajo futuro se mencionan varias veces a lo largo del documento. Estando además este trabajo orientado a futura investigación, continuar la optimización del código \textit{point-to-point} ha estado siempre en el horizonte cercano. Mediante el empleo de técnicas de \textit{Data Mining}, el uso de la herramienta MACVETH, así como mejorando la ubicación de los operandos en memoria para un mejor uso del ancho de banda de la memoria principal y la localidad caché, se espera una sustancial mejora en los rendimientos para índices de dispersión donde ya se supera a \texttt{OpenBLAS}, así como continuar expandiendo la viabilidad de esta aproximación para densidades mayores. En paralelo a estas labores más profundas de investigación y mejora de la herramienta, mejorarla es una prioridad, haciendo así que deje de ser una simple Prueba de Concepto. Esto se realizaría en varios pasos: 
\begin{itemize}
    \item Conversión del \textit{Jupyter Notebook} a una herramienta universal, programada en Python de inicio a fin, que lea una red neuronal como entrada y expulse código como salida. Esta herramienta podría ser tanto interactiva como no interactiva.
    \item Mejora en la modularización de la herramienta y generación de código. Ahora mismo los \textit{backends} implementados son engorrosos, lentos, y poco mantenibles, por lo que necesitarían una puesta a punto. Además, es conveniente dividir el código en varias secciones lógicas, para poder realizar la compilación y optimización en diversas \textit{translation units}\footnote{\url{https://en.wikipedia.org/wiki/Translation_unit_(programming)}}, y así aprovechar los múltiples núcleos que ofrece un procesador moderno.
    \item Implementar un mecanismo automatizado sobre esta herramienta mejorada, en la que se pueda seguir un \textit{workflow} ágil, pudiéndole suministrar enlaces a redes en formato \textit{ONNX} para la conversión a uno o más ficheros C, que sean analizados, optimizados, y compilados.
    \item Mejoría en la obtención de métricas, mejorando el actual sistema de medición de tiempos integrando, por ejemplo, un sistema de medición de contadores PAPI.
    \item Implementación de otros \textit{backends} para arquitecturas ya existentes como para arquitecturas \textit{ad hoc}, por ejemplo, en ensamblador o VHDL.
    \item Implementación de soporte para memoria distribuida mediante MPI\footnote{\url{https://www.mpi-forum.org/docs/}} o similar.
\end{itemize}


%%%%%%%%%%%%%%%%%%%%%%%%%%%%%%%%%%%%%%%%
% Apéndices, glosarios e bibliografía  %
%%%%%%%%%%%%%%%%%%%%%%%%%%%%%%%%%%%%%%%%

\appendix
\appendixpage
% \include{anexos/factsheet_europe}
% \include{anexos/bench_values}
% \include{anexos/archlinux_maintenance_guide}

\printglossary[type=\acronymtype,title=\nomeglosarioacronimos]
\printglossary[title=\nomeglosariotermos]

\bibliographystyle{IEEEtranN}
\bibliography{\bibconfig,bibliografia/bibliografia}
\cleardoublepage

\end{document}

%%%%%%%%%%%%%%%%%%%%%%%%%%%%%%%%%%%%%%%%%%%%%%%%%%%%%%%%%%%%%%%%%%%%%%%%%%%%%%%%

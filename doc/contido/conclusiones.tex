\chapter{Conclusiones}
\label{chap:conclusiones}

\lettrine{E}{n} este último capítulo se hace un balance general de este \acrlong{tfm}, a partir del cual se sacan conclusiones y se muestras posibles líneas de investigación futuras.

\section{Conclusiones}
El objeti

\section{Trabajo futuro}
Como se viene comentando a lo largo de todo el documento, hay múltiples líneas de trabajo con las que continuar. Estas posibles ``iteraciones'' adicionales deberían seguir en sí mismas un ciclo de vida incremental, dando así un ciclo de vida iterativo al propio Clúpiter, siendo esta la versión 1.0, por ejemplo.

En un ánimo de proponer mejoras o modificaciones que realizar en hipotéticas iteraciones futuras, se pueden encontrar:
\begin{itemize}
    \item Mejora del rendimiento de la CPU mediante.
    \item Mejora del ancho de banda de la memoria principal, tambien mediante.
    \item Mejora de la seguridad del sistema.
    \item Mejora y automatización de la mantenibilidad del sistema.
    \item Mejora del chasis, especialmente mediante el uso de una plegadora en el cortado de las planchas.
    \item Mejora del dashboard, preferiblemente extendiendo su funcionalidad y no modificándola.
\end{itemize}

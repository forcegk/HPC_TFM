\chapter{Introducción}
\label{chap:introducion}

\lettrine{E}{n} este capítulo se expondrá qué motivó este \acrlong{tfm} y sus objetivos, así como qué estructura seguirá la memoria del mismo.

\section{Motivación}
\label{sec:motivacion}
La computación de altas prestaciones (\acrshort{hpc}, o \acrlong{hpc}) y la \acrlong{ia} (\acrshort{ia}) están de moda. A pesar de que la primera es parte necesaria de muchos modelos complejos, que se mencionan más adelante en este trabajo, debido a la alta demanda de potencia de cómputo en la fase de entrenamiento de los mismos, es cierto que la inteligencia artificial es la más espectacular y tangible de cara al público.
Estos modelos de redes neuronales están evolucionando a enorme velocidad, y con ello sus posibles aplicaciones presentes y futuras, y debido a esta fuerte entrada de la \acrshort{ia} en lo \textit{mainstream}, son cada vez más necesarias soluciones de bajo consumo en una economía de escala.

\section{Objetivos}
\label{sec:objetivos}
Este trabajo supone un paso más en la nueva línea de investigación que comencé este curso académico, y por tanto no es de un muy elevado nivel debido a la falta de experiencia previa en este campo. Sin embargo es cierto que para esta memoria se han intentado incorporar conocimientos adquiridos en varias asignaturas cursadas durante este año en el Máster en Computación de Altas Prestaciones.



\section{Estructura}
\label{sec:estructura}
loremipsum
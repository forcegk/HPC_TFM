\chapter{Introducción}
\label{chap:introducion}

\lettrine{E}{n} este capítulo se expondrá qué motivó este \acrlong{tfm} y sus objetivos, así como qué estructura seguirá la memoria del mismo.

\section{Motivación}
\label{sec:motivacion}
La computación de altas prestaciones (\acrshort{hpc}, o \acrlong{hpc}) y la \acrlong{ia} (\acrshort{ia}) están de moda. A pesar de que la primera es parte necesaria de muchos modelos complejos, que se mencionan más adelante en este trabajo, debido a la alta demanda de potencia de cómputo especialmente en la fase de entrenamiento de los mismos, es cierto que la inteligencia artificial es la más espectacular y tangible de cara al público.
Estos modelos de redes neuronales están evolucionando a enorme velocidad, y con ello sus posibles aplicaciones presentes y futuras. Debido a esta fuerte entrada de la \acrshort{ia} en lo \textit{mainstream}, son cada vez más necesarias soluciones de bajo consumo en una economía de escala.

\section{Objetivos}
\label{sec:objetivos}
Este trabajo supone un primer paso en la nueva línea de investigación que comencé este curso académico, y por tanto no es de un muy elevado nivel debido a la falta de experiencia previa en este campo. Sin embargo, es cierto que para esta memoria se han intentado incorporar conocimientos adquiridos en varias asignaturas cursadas durante este año en el Máster en Computación de Altas Prestaciones.

\section{Estructura}
\label{sec:estructura}
Esta memoria describe el proceso de aprendizaje, conceptos básicos, y posibles líneas de mejora de la ejecución de modelos de inteligencia artificial en fase de inferencia. Dicho aprendizaje de conceptos, así como puesta en práctica se han realizado siguiendo un ciclo de vida en espiral, pero en esta memoria se muestran todas las fases comunes agrupadas.

De esta forma, primero se da una breve introducción y los conceptos básicos necesarios para comprender esta memoria. Tras ello, TODO xyz.
En el penúltimo capítulo se muestra la planificación e hipotéticos costes derivados de las horas invertidas en la investigación necesaria para la realización de este trabajo.

Todo esto para terminar con las conclusiones extraídas de este proceso, cómo este se relaciona con la titulación, así como las disciplinas requeridas para la realización del mismo. Las líneas de trabajo futuras también se resumen en estas, aunque de todos modos, este \acrshort{tfm} está ``orientado a líneas futuras'' principalmente.

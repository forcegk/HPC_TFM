\chapter{Conceptos básicos}
\label{chap:conceptos_basicos}

\lettrine{P}{ara} una correcta comprensión de los objetivos, tanto a corto como a largo plazo, de este trabajo, es necesario entender los conceptos básicos sobre los que se apoya esta línea de investigación.

\section{Redes Neuronales}
\label{sec:redes_neuronales}

La computación de altas prestaciones (\acrshort{hpc}, \acrlong{hpc}) y la inteligencia artificial (\acrshort{ia}) están de moda. Quizás más la segunda que la primera, pero lo cierto es que, sin base sobre la que ejecutarse, la segunda no tiene mucho que hacer.
Los supercomputadores son equipos informáticos compuestos por miles de procesadores, así como cantidades ingentes de memoria para ofrecer una elevada capacidad y velocidad de cálculo y procesamiento de datos.
Sin embargo, estas máquinas y sus mecanismos y procesos tan potentes suelen quedar muchas veces fuera de la comprensión no solo del público general, sino incluso de las personas que tienen la informática como \textit{hobby} o profesión.

Para no quedarse atrás, la Unión Europea está realizando una muy importante inversión en las dos disciplinas anterioremente mencionadas, pero especialmente en \acrshort{hpc}, donde la inversión ya supera a la de la \acrshort{ia}, como se puede ver en una hoja publicada por la Unión Europea en \cite{eu_factsheet_digital}.
Por esta razón, este \acrshort{tfg} tiene como objetivo atraer la atención de futuros ingenieros hacia este sector clave, que tanta inversión recibe y se espera que siga recibiendo. Para ello se construirá un pequeño clúster con Raspberry Pis con el nombre de Clúpiter, con el que poder realizar explicaciones acerca de cómo funciona este tan apasionante y demandado ámbito de la informática.

El principal componente de Clúpiter, la Raspberry Pi, es un computador de formato muy reducido y bajo consumo y coste, usado muy habitualmente en proyectos amateur y entornos educativos, lo que posibilita su empleo en proyectos de divulgación como el que aquí se realiza.
